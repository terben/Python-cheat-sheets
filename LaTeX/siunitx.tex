 \documentclass[12pt, a4paper]{scrartcl}

\usepackage[british]{babel}
\usepackage[utf8]{inputenc}
\usepackage[T1]{fontenc}
\usepackage{csquotes}
\usepackage{siunitx}
\usepackage{xspace}
\usepackage{booktabs}
\usepackage{listings}
\usepackage[framemethod=TikZ]{mdframed}
\usepackage{xcolor}
\usepackage{hyperref}

\lstset{language=[LaTeX]TeX}
\sisetup{range-phrase=--}

% definitions for framed boxes
\mdfdefinestyle{wrong}{%
  linecolor=red,
  roundcorner=5pt,
  linewidth=2pt,
  backgroundcolor=red!10}

\mdfdefinestyle{correct}{%
  linecolor=green,
  roundcorner=5pt,
  linewidth=2pt,
  backgroundcolor=green!10}

\newcommand{\siunitx}{\textit{siunitx}\xspace}

\newcounter{tip}[section] \setcounter{tip}{0}
\renewcommand{\thetip}{\arabic{section}.\arabic{tip}}
\newenvironment{tip}[2][]{%
\refstepcounter{tip}%
\ifstrempty{#1}%
{\mdfsetup{%
frametitle={%
\tikz[baseline=(current bounding box.east),outer sep=0pt]
\node[anchor=east,rectangle,fill=blue!20]
{\strut Practical Tip~\thetip};}}
}%
{\mdfsetup{%
frametitle={%
\tikz[baseline=(current bounding box.east),outer sep=0pt]
\node[anchor=east,rectangle,fill=blue!20]
{\strut Practical Tip~\thetip:~#1};}}%
}%
\mdfsetup{innertopmargin=10pt,linecolor=blue!20,%
linewidth=2pt,topline=true,%
frametitleaboveskip=\dimexpr-\ht\strutbox\relax
}
\begin{mdframed}[]\relax%
\label{#2}}{\end{mdframed}}

\begin{document}
%
\section{The \texttt{siunitx}-package to typeset
  numbers and units in \LaTeX{} documents}
Typesetting values with units correctly is a tedious but important,
and often overlooked, part of publication writing in the natural
sciences. It is particularly difficult for beginners to do it right
and unfortunately, \LaTeX{} does not offer much help for this task in
its standard configuration. Typesetting correct value/unit pairs is
non-intuitive and requires detail knowledge on microtypography.

Let us consider the following phrase which demonstrates common
issues and difficulties when dealing with values and units:
%
\begin{mdframed}[style=wrong,frametitle=values and units with errors]
  The -7 dB loss ($\pm 2 dB$) shown on pp.\,7-9 can be attributed to the
  $f(t) = \sin(2\pi ft)$ signal, where $t$ is the time and $f=48$\, Khz
  is the \emph{sampling frequency}.
\end{mdframed}
%
The phrase was typeset with the following \LaTeX{} source code:
%
\begin{mdframed}[style=wrong, frametitle=values and units with errors
  (\LaTeX{} source)]
  \begin{lstlisting}
  The -7 dB loss ($\pm 2 dB$) shown on pp.\,7-9 can be
  attributed to the $f(t) = \sin(2\pi ft)$ signal, where
  $t$ is the time and $f=48$\, Khz is the \emph{sampling
  frequency}.
  \end{lstlisting}
\end{mdframed}
%
If you are trained to spot typographical errors, you will discover the
following issues:
%
\begin{itemize}
  \item Similar to mathematical symbols in text, numbers must be typeset
    within a mathematical environment to be sure that they appear
    correctly. Outside math-mode, the \emph{minus sign} becomes the much
    shorter hyphen: -7 $\rightarrow$ $-7$. (\verb|-7| $\rightarrow$
    \verb|$-7$|).
    %
  \item Units of measurement are typeset outside math-mode in an
    upright font. Math-mode italic is used for mathematical variables,
    not for units and constants. Moreover, there must be a \emph{small
    space} between the value and the unit: $\pm 2 dB$ $\rightarrow$
    $\pm 2$\,dB (\verb|$\pm 2 dB$| $\rightarrow$ \verb|$\pm 2$\,dB|).
    %
  \item Number ranges are typeset with an en-dash, not with a hyphen:
    7-9 $\rightarrow$ 7--9 (\verb|7-9| $\rightarrow$ \verb|7--9|).
    %
  \item SI units have a standardized usage of uppercase versus
    lowercase letters. Prefixes kilo and below are abbreviated with a
    lowercase letter, mega and above with an uppercase letter. Symbols
    of base units names after a person start with an uppercase letter:
    Khz $\rightarrow$ kHz (\verb|Khz| $\rightarrow$ \verb|kHz|).
    %
  \item There must be a \emph{small space} (typeset in \LaTeX{} with
    \verb|\,|) between a value and its unit. Note than an additional
    space in the source code after \verb|\,| results in a wrong
    spacing: $f=48$\, kHz $\rightarrow$  $f=48$\,kHz
    (\verb|$f=48$\, kHz| $\rightarrow$ \verb|$f=48$\,kHz|).
\end{itemize}
%
Typeset correctly, the output and the source of the example phrase
look as follows:
\begin{mdframed}[style=correct,frametitle=values and units]
  The $-7$\,dB loss ($\pm 2$\,dB) shown on pp.\,7--9 can be attributed to the
  $f(t) = \sin(2\pi ft)$ signal, where $t$ is the time and $f=48$\,KHz
  is the \emph{sampling frequency}.
\end{mdframed}
%
\begin{mdframed}[style=correct,frametitle=values and units (\LaTeX{} source)]
  \begin{lstlisting}
  The $-7$\,dB loss ($\pm 2$\,dB) shown on pp.\,7--9 can be
  attributed to the $f(t) = \sin(2\pi ft)$ signal, where
  $t$ is the time and $f=48$\,kHz is the \emph{sampling
  frequency}.
  \end{lstlisting}
\end{mdframed}
%
It is obvious that a \LaTeX{} package, which takes away from the author
much of the burden with this kind of microtypography, is very desirable.

\siunitx is such a package. It is very powerful \LaTeX-package and
provides you with a consistent syntax to typeset values, ranges,
lists, units, tabulated data, and uncertainties. It also handles
country-specific typographic rules and it is highly configurable. In
the following, we highlight some of its features with short examples.
For a comprehensive manual, please see the well written documentation
at \url{http://tug.ctan.org/macros/latex/exptl/siunitx/siunitx.pdf}.
%
\subsection{The basic value-unit pair}
In \siunitx, one types a value with a unit using the \verb|\SI|
command. This command works in both math-mode and inline with text. In
both environments the value and the unit are rendered in the same
way. The syntax is fairly self-explanatory. The package supports a
very large number of different units and differewnt ways to type them.
For beginners, the most intuitive way is to typeset units
\emph{literally} such as \verb|\tera\electronvolt| or
\verb|\centi\meter|. For a full list of available units check-out the
\siunitx documentation. You can also typeset composite units such as
\verb|\centi\meter\per\second| and so on. \siunitx automatically takes
care of correct spacing between units. Table~\ref{tab:numb_units}
gives some examples.
%
\begin{table}[ht]
  \caption{Some basic \siunitx-examples}
  \label{tab:numb_units}
  \centering
  \begin{tabular}{lcp{4cm}}
    \toprule
    \textbf{\LaTeX-input} & \textbf{text output} & \textbf{comments} \\
    \midrule
    \verb|\SI{8}{\kilo\meter}| & \SI{8}{\kilo\meter} & simple unit \\
    \verb|\SI{8}{\kilo\meter\per\second}| &
      \SI{8}{\kilo\meter\per\second} & composed unit \\
    \verb|\SI{8.0e03}{\meter\second}| &
      \SI{8.0e03}{\meter\second} & numbers can be typeset in many
      different formats (here in raw program output) \\
    \verb|\SI{0.8e05}{\milli\second}| &
      \SI{0.8e05}{\milli\second} & note the different settings of the
      units \verb|\meter\second| (last example) and
      \verb|\milli\second|. The first one requires a small space
      between \texttt{m} and \texttt{s}!\\
    \verb|\SI{245.6 +- 10}{\meter\squared}| &
      \SI{245.6 +- 10}{\meter\squared} & numbers with errors \\
    \midrule
    \verb|\ang{10;5;2}| & \ang{10;5;2} & angles \\
    \verb|\num{123456789}| &
      \num{123456789} & long numbers get correct grouping  \\
    \verb|\num{1 + 2i}| &
      \num{1+2i} & complex numbers \\
    \midrule
    \verb|\si{\tera\electronvolt}| &
      \si{\tera\electronvolt} & units without associated numbers  \\
    \midrule
    \verb|\SIrange{2}{10}{\percent}| &
      \SIrange{2}{10}{\percent}  & ranges of numbers  \\
    \midrule
    \verb|\SIlist{2;9;10}{\cm}| &
        \SIlist{2;9;10}{\cm} & lists of numbers  \\
    \bottomrule
  \end{tabular}
\end{table}

If you need to render units or values on their own, \siunitx provides
the \verb|\si| and \verb|\num| commands. The former will typeset large
numbers with a delimiter to group digits together to improve
readability. \siunitx also offers possibilities to typeset numbers
with errors, ranges of numbers and so on. Table~\ref{tab:numb_units}
gives a few examples but you need to consult the
\siunitx-documentation for any details.

With \siunitx, the example phrase would be typeset and output as
follows:
%
\begin{mdframed}[style=correct,frametitle=values and units with
  \siunitx (\LaTeX{} source)]
  \begin{lstlisting}
  The \SI{-7}{\deci\bel} loss (\SI{+- 2}{\deci\bel}) shown on
  pp.\,\numrange{7}{9} can be attributed to the
  $f(t) = \sin(2\pi ft)$ signal, where $t$ is the time and
  $f=\SI{48}{\kilo\hertz}$ is the \emph{sampling frequency}.
  \end{lstlisting}
\end{mdframed}
%
\begin{mdframed}[style=correct,frametitle=values and units with \siunitx]
  The \SI{-7}{\deci\bel} loss (\SI{+- 2}{\deci\bel}) shown on
  pp.\,\numrange{7}{9} can be attributed to the
  $f(t) = \sin(2\pi ft)$ signal, where $t$ is the time and
  $f=\SI{48}{\kilo\hertz}$ is the \emph{sampling frequency}.
\end{mdframed}
%
\begin{tip}{tip:unit_typeset}
  Many people think that they are sufficiently comfortable with the
  necessary typographical rules. They quickly tend to not typeset units
  literally like
  \SI{1}{\tera\electronvolt} (\verb|\SI{1}{\tera\electronvolt}|)
  but to do it \emph{shorter} such as \SI{1}{TeV} (\verb|\SI{1}{TeV}|).

  I strongly advice you not to do so! You give away most of
  the assistance to typeset units correctly. Do you note the
  difference between \SI{1}{\meter\newton}
  (\verb|\SI{1}{\meter\newton}|) and \SI{1}{\milli\newton}
  (\verb|\SI{1}{\milli\newton}|)? How would you typeset those units
  without the literal approach?
\end{tip}
%
\end{document}
